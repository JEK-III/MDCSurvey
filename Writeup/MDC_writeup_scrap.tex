% ==============================================================
% Information about data users
% ==============================================================

Without question, the most invested consumer of data metrics are dataset creators themselves; metrics can provide a sense of what data is being used for, and reassure them that it is being used and that sharing is worthwhile.
Secondarily, data managers have a stake in knowing about users of their data to justify funding and tailor services.
We asked what researchers would be most interested to know about users of their data and, not surprisingly, many wanted to know everything: researchers chose ``name and contact information'' as their first choice than any other option (48\%).
However, a significant fraction (13\%) ranked this option last (of five), and the highest average interest in discipline was slightly higher.
The least interesting option we presented was geographic location, put in last place by 58\%.
Data managers overwhelmingly chose discipline as the most interesting thing to know about users of their data (67\%).
The current practices of repositories were split roughly in half between repositories that require a name (47\%), email address (44\%), and institutional affiliation (40\%) and those that provide data without collecting any information (47\%).


but it is more surprising that data managers prefer a purely scholarly measure so clearly over ``real world'' impact (ranked first by only 23\%). 

% Carly:
% I wanted some kind of wrap-up sentence/paragraph here. maybe bring it back to the DLM project? how we plan to use these data?

* These results provide clear guidance for projects

your email, please do clarify the relation of this work to the survey published recently at PLOS One.  We feel that it is important that this be seen as a new contribution on the topic, which complements this earlier study, not an addendum on the previous work.  

When we send this to our board for formal commenting and feedback, we will need to be able to grant them access to the underlying survey data, in the form you intend to release to the public with final publication.  Will this be possible using your chosen CDL repository?  If needed we can host the data temporarily as supplementary material.

Since this is being considered for publication at Scientific Data, we felt it may be appropriate to include a sentence or two mentioning our journal's use of data citations.  We were the first NPG journal to include formal data citations in all articles, and our formatting was designed specifically to meet the principles laid out in the Joint Declaration, which was signed by NPG. Here is a related blog post on the topic (http://blogs.nature.com/scientificdata/2014/03/24/endorsing-the-joint-declaration-of-data-citation-principles/). NPG staff have also been involved directly in ongoing updates to JATS that are designed to help accommodate data citation, particularly Paul Donohoe (https://www.force11.org/user/8019).  

Unfortunately, our Comment article format does not include a dedicated Data Citation section (the irony does not escape me).  It is fine though to cite the data record within the main references section.  
