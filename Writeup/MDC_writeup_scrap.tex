% ==============================================================
% Information about data users
% ==============================================================

Without question, the most invested consumer of data metrics are dataset creators themselves; metrics can provide a sense of what data is being used for, and reassure them that it is being used and that sharing is worthwhile.
Secondarily, data managers have a stake in knowing about users of their data to justify funding and tailor services.
We asked what researchers would be most interested to know about users of their data and, not surprisingly, many wanted to know everything: researchers chose ``name and contact information'' as their first choice than any other option (48\%).
However, a significant fraction (13\%) ranked this option last (of five), and the highest average interest in discipline was slightly higher.
The least interesting option we presented was geographic location, put in last place by 58\%.
Data managers overwhelmingly chose discipline as the most interesting thing to know about users of their data (67\%).
The current practices of repositories were split roughly in half between repositories that require a name (47\%), email address (44\%), and institutional affiliation (40\%) and those that provide data without collecting any information (47\%).


but it is more surprising that data managers prefer a purely scholarly measure so clearly over ``real world'' impact (ranked first by only 23\%). 

