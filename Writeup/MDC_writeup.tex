\documentclass[english]{article}
\usepackage[T1]{fontenc}
\usepackage[latin9]{inputenc}
\usepackage{babel}

\usepackage{graphicx}
\usepackage[hidelinks]{hyperref}

% Use the PLoS provided bibtex style temporarily
\bibliographystyle{plos2009}

\begin{document}

\title{Making Data Count}


\author{John Kratz\textsuperscript{1{*}}, Carly Strasser\textsuperscript{2}}

\maketitle
1. California Digital Library 2. DataCite 
{*}corresponding author(s): John Kratz (John.Kratz@ucop.edu)


\section*{Comment}

% ==============================================================
% Introduction
% ==============================================================

Data undergirds all of science and should be recognized as valuable scholarship, which requires an accepted set of metrics to gauge the contribution made by any given dataset.
An article's significance has long been estimated by counting the number of subsequent articles that cite it. 
Now, a suite of internet-based alternative metrics (``altmetrics'') seek to provide faster assessment and capture other kinds of impact\cite{priem_altmetrics_2012}.
Data descriptor articles published in journals like \textit{Earth Systems Science Data} or \textit{Scientific Data} could serve as traditionally-measurable proxies for datasets\cite{pfeiffenberger_earth_2008, editors_more_2014}.
Datasets themselves are susceptible to many of the same metrics as articles-- but the translation is not straightforward.
The activities of viewing a landing page, downloading, or citing are all likely to have different significance in relation to a dataset than an article.

Multiple efforts are underway investigate traditional, alternative, and novel metrics for data: the Research Data Alliance (RDA) Data Bibliometrics Working Group (\url{http://rd-alliance.org/group/rdawds-publishing-data-bibliometrics-wg/case-statement/rdawds-publishing-data-bibliometrics-wg}{}); the National Information Standards Organization (NISO) Alternative Assessment Metrics Initiative (\url{http://www.niso.org/topics/tl/altmetrics_initiative/}), which explicitly considers altmetrics for non-traditional products like software and data; and Making Data Count (\url{http://mdc.plos.org}), a collaboration between the California Digital Library, the Public Library of Science (PLOS), and DataONE to define a suite of Data Level Metrics and adapt the existing PLOS Article Level Metric (ALM) tool(\url{http://alm.plos.org}) to capture them.

% ==============================================================
% Methods and Demographics
% ==============================================================

To define useful metrics, we must understand the values and needs of the researchers who create and re-use data, and of the data managers who preserve and publish it.
During the research phase of Making Data Count, we asked researchers and database or repository staff questions about data sharing, discovery, and metrics.
We solicited responses to a pair of online surveys via social media, listservs, and posts to CDL and PLOS blogs-- ultimately hearing from 247 researchers and 73 data managers in November and December of 2014.
Most researchers worked at academic institutions (78\%), and most were based in the United States (57\%) or United Kingdom (14\%).
Both academic (64\%) and government-run (22\%) repositories were significantly represented; repositories were primarily located in the United States (72\%) or United Kingdom (11\%).
Researchers from across the academic career spectrum responded; principal investigators (42\%), postdocs (21\%), and graduate students (19\%) were all well represented. 
Biology is the most popular discipline (53\%), but environmental (17\%) and social (10\%) science are also significantly represented. 

% ==============================================================
% Locations for sharing & discovery
% ==============================================================

To situate the collection and display of data metrics, we wanted to know where on the internet researchers go to share their data and to find data to use. 
Data sharing behavior has been relatively well-surveyed\cite{tenopir_data_2011, akers_disciplinary_2013, wallis_if_2013, aydinoglu_data_2014, kratz_researcher_2015}. 
Consistent with these previous surveys, direct transmission (e.g., via email) on request was the most common behavior:74\% of respondents had shared some data this way and 16\% had shared ``most/all'' of their data that way.
Among the drawbacks to this approach-- including the potential for capricious denial of access-- is that it is invisible to measurement. 
Fortunately, a more tractable method, publication via a database or repository, is also widely used.
It is the most common channel for sharing most/all of the data (used by 24\%), presumably because database upload is more likely to be systematic than responding to particular requests; another 47\% shared some of their data that way.

<<<<<<< Local Changes

% ==============================================================
% Information about data users
% ==============================================================

A major factor encouraging data sharing on request is researcher's desire to know who is using their data for what purpose.
Fear of misuse is a major concern \cite{kuula_open_2008, kuipers_insight_2009}.
Data metrics or aggregation of information about data use could potentially satisfy that interest for data in public repositories.
We asked what researchers would be most interested to know about users of their data and, not surprisingly, many wanted to know everything.
The most popular option, ``name and contact information,'' was chosen as the most interesting by 48\% of researchers, much more than the less-specific options: discipline (32\%), institution type (8\%), institutional affiliation (6\%), and geographic location (5\%).
However, a significant fraction (13\%) ranked this option last, and the average interest in discipline was equally high.
When asked a similar question, data managers overwhelmingly chose discipline as the most interesting thing to know about users of their data (67\%).
In terms of current practice, repositories are split roughly in half between that collection extensive information-- a name (47\%), email address (44\%), and institutional affiliation (40\%)-- and not collecting any (47\%).


% ==============================================================
% Discovery & use
% ==============================================================

Compared to how scientists share data they have created, how and where they search for data in their role as (re-)users is not well understood.
We asked researchers how likely they would be to use each of five possible strategies (shown in Figure \ref{fig:results}a).
=======
How and where scientists search for data in their role as (re-)users has been less well studied, so we asked researchers how likely they would be to use each of five possible strategies (Figure \ref{fig:results}a).
>>>>>>> External Changes
No single method predominated.
A 63\% majority said that they would ``definitely'' use more than one strategy, and three strategies would definitely be used by most respondents: searching via references in the literature (59\%), a discipline-specific database (58\%), or a general purpose search engine (51\%). 
When asked to name particular sources, the general-purpose repository Dryad (\url{http://datadryad.org/}) was most frequently mentioned ($n=16$); Google and ``journal articles'' were tied for second ($n=14$). 
Respondents were unlikely to `crowd-source' data discovery via open inquiries on social media (42\% ``no chance'') or discussion forums (40\% no chance).
However, direct inquiries to knowledgeable colleagues are probably more common, and were a relatively highly cited write-in source ($n=12$).

We sought to characterize the role of public data in respondents research process by asking how frequently they used public data to generate ideas/hypotheses at the outset, to reach the main conclusion of a paper, or to support the main conclusion of a paper (Figure \ref{fig:results}b).
An overwhelming 96\% of respondents said that they at least ``occasionally'' used public data in at least one capacity; a smaller 56\% majority used public data ``often'' in at least one capacity.
Furthermore much of this use is central to research, 28\% often and 42\% occasionally use public data to reach the main conclusions of a paper, although use for idea generation (41\% often, 46\% occasionally) and support (39\% often, 47\% occasionally) were higher.

% ==============================================================
% Information about data users
% ==============================================================


We asked what researchers would be most interested to know about users of their data and, not surprisingly, many wanted to know everything: researchers chose ``name and contact information'' as their first choice than any other option (48\%).
However, a significant fraction (13\%) ranked this option last (of five), and the highest average interest in discipline was slightly higher.
The least interesting option we presented was geographic location, put in last place by 58\%.
Data managers overwhelmingly chose discipline as the most interesting thing to know about users of their data (67\%).
The current practices of repositories were split roughly in half between repositories that require a name (47\%), email address (44\%), and institutional affiliation (40\%) and those that provide data without collecting any information (47\%).

% ==============================================================
% Information about data impact 
% ==============================================================

Without question, the most invested consumer of data metrics are dataset creators themselves; metrics can provide a sense of what data is being used for, and reassure them that it is being used and that sharing is worthwhile.
Secondarily, data managers have a stake in knowing about users of their data to justify funding and tailor services.
We asked both groups what impact metrics they would most be interested in knowing about their data. 

For both groups, scholarly prestige is still measured in citations.
When asked to rank potential information about data use, 85\% of researchers (Figure \ref{fig:results}c) and 61\% of data managers chose citations as the most interesting thing to know about their data. 
<<<<<<< Local Changes
The preference of researchers is consistent with previous surveys\cite{kratz_researcher_2015}.
=======
The preference of researchers is consistent with previous surveys\cite{@kratz_researcher_2015}, 
>>>>>>> External Changes
Download count was a consistent second choice of researchers (by 64\%). 
Landing page views were ranked last by >50\% of both researchers and data managers.

To 
As a practical matter, it is important to understand what metrics are already being tracked and exposed by repositories(Figure \ref{fig:results}d). 
We found that most repositories track downloads (85\%) and landing page views (66\%). 
However, only 35\% of the repositories that track downloads expose them via API or display on the landing page; this ratio is roughly similar for all of the metrics we asked about. 
Despite the extreme interest in citation counts, relatively few repositories (23\%) track them. 

% ==============================================================
% Conclusions & future work 
% ==============================================================

<<<<<<< Local Changes
While citation counts are the gold standard, their current usefulness suffers from the major limitation that datasets are rarely cited formally.
The clearest illustration of current practice is a 2011 survey of social science papers that found that only 17\% cited the dataset with at least the title in the reference list, roughly the same as in 1995 
\cite{sieber_not_1995, mooney_citing_2011}. 
However, there are reasons for optimism.
Data creators strongly endorse formal data citation: 95\% of respondents to a 2011 DataONE survey agreed that formal citation was a fair condition for data sharing, as did 87\% of astrobiologists in a follow-up survey \cite{tenopir_data_2011, aydinoglu_data_2014}. 
Citation ``in the references like normal publications'' is the preferred method of receiving credit for data sharing by 71\% of biodiversity researchers and by 75\% of respondents to our earlier survey \cite{enke_users_2012, kratz_researcher_2015}.
In 2014, the scholarly communication community recognized the consensus that data used in an article should be cited formally in the reference list with the Joint Declaration of Data Citation Principles (\url{http://www.force11.org/node/4769}), and a number of organizations like DataCite, Future of Research Communication and e-Scholarship, and the Research Data Alliance are actively working to overcome the technical and cultural obstacles to widespread data citation.

Unlike citations, repositories can easily track dataset landing page views and downloads.
Page views considered to be of little value by researchers and data managers, but downloads are more highly regarded; it stands to reason that downloading a dataset represents a higher level of engagement that simply viewing the landing page.
For researchers, downloads were a very consistent second-choice to citations.
Furthermore, our previous work suggests that although the preference for citations is consistent, the gap in perceived value between the two metrics is relatively narrow \cite{kratz_researcher_2015}.
Downloads are already widely tracked by repositories, and we strongly recommend that more repositories make them public.
=======
Citation counts are the gold standard, but there are two major obstacles to citation counts: one is that data is often not cited formally; the other is that the infrastructure to track data citations isn't entirely in place.
The scholarly communication community recognized the consensus that data used in an article should be cited formally in the reference list with the Joint Declaration of Data Citation Principles (\url{http://www.force11.org/node/4769}). 
Researchers agree: 95\% of respondents to a DataONE survey agreed that formal citation was a fair condition for data sharing, as did 87\% of astrobiologists in a follow-up survey \cite{@tenopir_data_2011, @aydinoglu_data_2014}. 
Citation``in the references like normal publications'' is the preferred method of receiving credit for data sharing by 71\% of biodiversity researchers and by 75\% of respondents to our earlier survey \cite{@enke_users_2012, @kratz_researcher_2015}.
The Thomson-Reuters Data Citation Index (\url{http://thomsonreuters.com/data-citation-index/}) is a good start, but we need to firm up this infrastructure.
% something about citation harvesting infrastructure
% maybe something about data journals

Download counts and page views are much more manageable.
Downloads are considered valuable.
Page views aren't.
Repositories should collect and expose DLs at least.

>>>>>>> External Changes


\section*{Acknowledgements}

Making Data Count is funded by National Science Foundation (NSF) grant number 1448821.

AK did this and that. 

BG did this and that and the other. 


\section*{Competing financial interests}


The author(s) declare no competing financial interests.


\section*{Figures Legends}

Figure should be referred to using a consistent numbering scheme through
the entire Data Descriptor. For initial submissions, authors may choose
to supply this document as a single PDF with embedded figures, but
separate figure image files must be provided for revisions and accepted
manuscripts. In most cases, a Data Descriptor should not contain more
than three figures, but more may be allowed when needed. We discourage
the inclusion of figures in the Supplementary Information \textendash{}
all key figures should be included here in the main Figure section. 

Figure legends begin with a brief title sentence for the whole figure
and continue with a short description of what is shown in each panel,
as well as explaining any symbols used. Legend must total no more
than 350 words, and may contain literature references. 

\begin{figure}[!ht]
\begin{center}
\includegraphics[width=6in]{MDC_Figure_rough_sketch.png}
\end{center}
\caption{
{\bf Citations are valuable}
Additional explanation.
Error bars depict bootstrapped 95\% confidence intervals. 
}
\label{fig:results}
\end{figure}

\begin{thebibliography}{1}
\expandafter\ifx\csname url\endcsname\relax
  \def\url#1{\texttt{#1}}\fi
\expandafter\ifx\csname urlprefix\endcsname\relax\def\urlprefix{URL }\fi
\providecommand{\bibinfo}[2]{#2}
\providecommand{\eprint}[2][]{\url{#2}}

\bibitem{cite1}
\bibinfo{author}{Califano, A.}, \bibinfo{author}{Butte, A.~J.},
  \bibinfo{author}{Friend, S.}, \bibinfo{author}{Ideker, T.} \&
  \bibinfo{author}{Schadt, E.}
\newblock \bibinfo{title}{{Leveraging models of cell regulation and GWAS data
  in integrative network-based association studies}}.
\newblock \emph{\bibinfo{journal}{Nature Genetics}}
  \textbf{\bibinfo{volume}{44}}, \bibinfo{pages}{841--847}
  (\bibinfo{year}{2012}).

\bibitem{cite2}
\bibinfo{author}{Wang, R.} \emph{et~al.}
\newblock \bibinfo{title}{{PRIDE Inspector: a tool to visualize and validate MS
  proteomics data.}}
\newblock \emph{\bibinfo{journal}{Nature Biotechnology}}
  \textbf{\bibinfo{volume}{30}}, \bibinfo{pages}{135--137}
  (\bibinfo{year}{2012}).

\end{thebibliography}

\bibliography{Comment}

\section*{Data Citations}

1. Kratz, J., \& Strasser, C.,  \emph{UC Office of the President} DOI (2015).

\end{document}

